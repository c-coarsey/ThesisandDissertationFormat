\documentclass[phd, 12pt, print]{fauthesis}

% The "print" class option overrides any special formatting of links by the hyperref package.
% Use the "online" class option instead to color links and/or put boxes around them.


%% The hyperref package embeds document data in PDF files, and automatically creates PDF bookmarks for chapters and sections.  The following lines activate it and set some of its options.
%
% NOTE: The "print" class option above will override the options used here to color the links.  Change the class option to "online" to avoid this.
%%
\usepackage{hyperref}

\hypersetup{
	pdftitle={fauthesis Sample File},						% title
	pdfauthor={C. Beetle},								% author
	pdfsubject={Sample file for the fauthesis document class.},	% subject of the document
	pdfnewwindow=true,								% links in new window
	pdfkeywords={FAU, thesis, dissertation, guide, 
					TeX, LaTeX, class, style, format},		% list of keywords
	colorlinks=true,										% false: boxed links; true: colored links
	linkcolor=[rgb]{0.7,0.0,0.0},							% color of internal links
	citecolor=[rgb]{0.0,0.4,0.4},							% color of links to bibliography
	filecolor=[rgb]{0.0,0.0,0.8},							% color of file links
	urlcolor=[rgb]{0.0,0.0,0.8}								% color of external links
}


\title{A Remarkably Brief and Rather Incomplete\\ History of Florida Atlantic University}
\author{Anne Owl Burroughs}
\gender{F}
\graduation{December}{2014}
\department{Department of Physics}
\college{Charles E.~Schmidt College of Science}
\chair{Warner A.~Miller}
\dean{Russell Ivy}
\deantitle{Interim Dean}
\dgc{Deborah L.~Floyd}[Ed.D.]

\advisor{Christopher Beetle}
\supervisor{Andy Lau}
\supervisor{Theodora Leventouri}
\supervisor{Korey Sorge}
\supervisor{Wolfgang Tichy}


\signaturepush{-2.0}										% 	Increase or decrease the vertical gap between committee and administration signatures


\begin{document}

\frontmatter

\maketitle
\makecopyright
\makesignature

\begin{vita}													% Optional
Anne Owl Burroughs is an invention of the fevered imagination of her thesis advisor.  Any resemblance to persons living or dead would be quite a shock.
\end{vita}

\begin{acknowledgements}
The author has stolen the entire text of this sample document from the official website of Florida Atlantic University \cite{HFAU}.  The sample document is meant to illustrate how the \texttt{fauthesis} class is used.
\end{acknowledgements}

\begin{abstract}
Florida Atlantic University was established by the Florida State Legislature in 1961 as the fifth university in the state system.  When it opened its doors in 1964, FAU was the first university in the country to offer only upper-division and graduate-level work.  This model was based on the theory that freshmen and sophomores would be served by the community college system.  In 1984, the University responded to population growth and the need to provide increased access to higher education by admitting its first freshmen class.  Today, FAU's seven partner campuses serve 26,000 students through more than 170 degree programs.  FAU's colleges include the College of Architecture, Urban \& Public Affairs, the Dorothy F.~Schmidt College of Arts \& Letters, the Barry Kaye College of Business, the College of Education, the College of Engineering, the Christine E.~Lynn College of Nursing, the Charles E.~Schmidt College of Science, the Charles E.~Schmidt College of Biomedical Science and the Harriet L.~Wilkes Honors College, which provides a unique and challenging four-year curriculum for the brightest students from Florida and beyond.  FAU has Eminent Scholar Chairs in many academic disciplines, and it is the home of nationally recognized research centers.  The University's burgeoning research parks are facilitating exciting new research and learning initiatives by bringing high-tech industries into close collaboration with faculty and students.
\end{abstract}

\dedicate{To the graduate students of Florida Atlantic University.} 		% Optional

\tableofcontents
\nolistoftables					% Or \listoftables if there are tables
\listoffigures					% Or \nolistoffigures if there are no figures


\mainmatter


\chapter{Prehistory}

On a bright October day in 1964, Lyndon Baines Johnson, $36^{\mathrm{th}}$ President of the United States, squinted into the South Florida sun and, in his famous Texas drawl, declared Florida Atlantic University officially open.

For a sitting U.S.\ chief executive to officiate the dedication of a new regional university was most unusual  ---  but, then, FAU was no ordinary institution of higher learning.  From its very inception, FAU was envisioned as the first of a new breed of American universities that would quite deliberately throw off the ivy-covered trappings of the tradition-bound world of academe and invent new and better ways of making higher education available to those who sought it.

Indeed, in his dedication remarks, President Johnson said that America had entered an era ``when education is no longer only for the sons of the rich, but for all who can qualify.''  Speaking on an outdoor stage before a crowd of 15,000, he called for ``a new revolution in education'' and said that a fully educated American public could vastly enrich life over the next 50 years.

Seated onstage behind the President as he spoke was an array of Florida's top political VIPs, including Governor Farris Bryant, U.S.\ Senators Spessard Holland and George Smathers, U.S. Congressmen Claude Pepper and Paul Rogers, and a banker named Thomas F. Fleming, Jr., who, more than anybody else, was responsible for bringing America's newest public university to Boca Raton.

\section{From Airbase to Campus}

In the beginning, there was an airbase  ---  the Boca Raton Army Air Field, to be exact.  This facility, one of the few radar training schools operated by the U.S.\ Army Air Corps during World War Two, opened in October 1942 in the sleepy coastal resort town of Boca Raton.  The base, which eventually covered more than 5,800 acres, did its part to help win the war, teaching the relatively new art of radar operation to thousands of airmen, including those who were aboard the Enola Gay on its fateful run to Hiroshima in 1945.  By the 1950s, however, the base had outlived its usefulness; the radar training school it once housed had moved to Biloxi, Mississippi, and weeds grew tall around the landing strips that once saw a steady stream of arriving and departing B-17 and B-29 bombers.  The war was over, and America was facing new challenges, including the imminent coming of age of the first wave of Baby Boomers.  Members of the most economically privileged generation in U.S.\ history, they were going to seek higher education in record numbers, and Florida's colleges and universities were in no way prepared for the onslaught.

In 1955, the Florida Legislature authorized creation of a new public university to serve the populous southeast region of the state.  The new university would be the fifth in the State University System, joining the University of Florida in Gainesville, Florida State University and Florida A\&M University in Tallahassee, and the University of South Florida in Tampa. Community leaders in Broward and Palm Beach counties stepped forward to suggest possible sites, none with more enthusiasm than Boca Raton's Tom Fleming, who made a convincing case for converting the vacated airbase to this exciting new use.

Fleming was a true visionary who recognized the many benefits a state university had to offer Boca Raton.  The son of a prominent Fort Lauderdale attorney and bank president, he had arrived in Boca Raton in 1941 to help manage the 4,000-acre Butts Farm, which was owned by the family of his wife, Myrtle, and he often referred to himself as ``a bean farmer.''  His educational credentials included a bachelor's degree from the University of Florida, where he had been a member of the prestigious Blue Key leadership honorary society, and an MBA from Harvard.

Tom Fleming was successful at everything he did, and everywhere he went he made influential friends.  By the time he was heading up the drive to establish the new state university in Boca Raton  ---  under the rallying cry of �Boca U.\ in '62�  ---  he had many friends in Tallahassee and Washington who would prove to be powerful allies.

On January 18, 1957, Fleming stood before the Board of Control, which was the body that governed public universities in Florida at that time, and presented his proposal.  When one member objected that the 400 feet of beachfront property owned by the city was insufficient to accommodate large groups of collegians, another member replied: ``We want to educate them, not give them a bath.''  By meeting's end, the Board had unanimously endorsed Fleming's idea, disappointing proponents of the other proposed sites.

Next came complex negotiations in Washington to get the federal government to lift use restrictions off the land.  Ultimately, the Civil Aeronautics Administration agreed to permit the state to build the university on 1,000 acres of the former airbase, reserving another 200 acres for airport use.  Boca Raton Municipal Airport was built on a 200-acre site adjoining the campus and remains in active use to this day.

In 1960, the State Cabinet, sitting as the Board of Education, gave final approval to the Boca Raton site.  The new university's opening date was set for September 1964.

\section{``Open the Door in '64''}

Just one hitch remained: while the state had approved building a new university in Boca Raton, it had provided no funding for planning, architectural design or construction.  When Broward Culpepper, chairman of the Board of Control, announced that the local community would have to raise \$100,000, Fleming swung into action once again, establishing an Endowment Corporation that solicited contributions from the public under the slogan ``Open the Door in '64.''

The first donation came from Fleming himself, who pledged one percent of three years' worth of the pre-tax earnings of the First Bank and Trust Company of Boca Raton, which he headed.  The Endowment Corporation raised close to \$300,000 in start-up funding for the university, and it is still in service today under the name of the FAU Foundation.

Next came the question of what to name the new university.  There was no lack of ideas from official quarters or the public.  Names generated through a contest run by the Fort Lauderdale News included Palm State, Peninsula University, Gulfstream University, Kennedy University of Florida, Bryant State (to honor Governor Farris Bryant, a Fleming friend who was an early supporter of the Boca site), Sunshine State and A-Okay University (a reference to a catch-phrase used in the 1960s by American astronauts).  The Board of Control resolved the question by adopting the name Florida Atlantic University in 1962, two years before the scheduled opening.

Tom Fleming made a critically important discovery during his long, successful campaign to bring FAU into existence: He realized that state support of all of higher education in Florida was woefully inadequate.  In order to remedy this, he became chairman of ``Citizens for Florida's Future,'' a committee of the state Chamber of Commerce that sought voter approval of a \$75 million bond issue to expand and improve Florida's junior colleges and universities.

The bond issue passed in the November 1963 election, and President John F.\ Kennedy praised Fleming by name for this outstanding accomplishment during a speech that month in Tampa.  It was the last speech Kennedy made before his tragic trip to Texas.  A letter inviting him to take part in the planned dedication of FAU the following year was mailed on the very day he was assassinated in Dallas.

By the time FAU was ready to open in the fall of 1964, Lyndon Johnson was President, and he was campaigning hard against Republican candidate Barry Goldwater.  Fleming, who was managing Johnson's Florida campaign, made him an offer he couldn't refuse: He asked him to make the keynote address at the ceremony that would mark the opening of Florida's newest public university.  And that is why the President of the United States was on hand when Florida Atlantic University was dedicated.

\section{A New Kind of University}

From the very beginning, FAU aimed to be a whole new kind of university, one that would harness broadcast technology to beam classes to students wherever they might be, thus swinging the door of higher education open wider than ever before.  In a very real sense, FAU was the first Information Age university.  The only problem was that the Information Age itself would not be popularly recognized for nearly four more decades, and the outside technology needed to enable off-campus students to take advantage of what FAU had to offer simply did not exist.  It finally began to take shape in the late 1990s, as colleges and universities around the world started offering increasing numbers of courses online and through other methods of distance learning.

The first university buildings to rise among the quonset huts of the old airbase were the Library, the Learning Resources Building, the Sanson Science Building and General Classrooms South, which featured classrooms shaped like slices of pie arranged around a core containing the most advanced audio-visual resources available in the early 1960s.  A few steps away, in Learning Resources, four fully equipped television studios stood ready to broadcast classroom lectures across campus or around the world.  The Library featured a technologically sophisticated Media Center, an automatic check-out system and a computer-generated catalog instead of the familiar Dewey Decimal System card index.

FAU was the first university in the country to offer only upper-division and graduate-level work, on the theory that freshmen and sophomores could be served by the growing community college system.  Even with these enrollment restrictions, the initial student body was expected to be about 2,000, but by September 8, 1964, the scheduled opening day, fewer than half that number had registered for classes.  This shortfall was attributed to the campus' lack of dormitories and dining facilities, South Florida's inadequate system of highways, the absence of public transportation and the administration's failure to actively recruit students.  Because a feasibility study had indicated that the new university stood in the middle of a region that was home to 30,000 potential students, little or no marketing effort had been made.

Just as FAU was about to open, Hurricane Cleo swept its way up Florida's east coast, causing \$100,000 in damage to the campus and delaying the start of classes by six days.  When the wind died down and the flood waters receded, FAU's charter class of 867 students arrived to begin their studies on a treeless campus marked by a flagpole that was bent like a used pipecleaner.  Thus did the academic life of the university get under way, inspired by the motto ``Where Tomorrow Begins.''


\chapter[The Past Presidents of Florida Atlantic University]{The Past Presidents of\\ Florida Atlantic University}

\section{The Williams Years (1962--1973)}

During its early years, FAU prospered and grew, led by the steady hand of its first president, Dr.\ Kenneth R.\ Williams, who had also been the founding president of Miami-Dade Community College (then called Dade County Junior College).  In 1962, two years before the university opened, he took the helm of an institution with a small but dedicated faculty that quickly became known for outstanding classroom teaching and mentorship of students.  These qualities continue to characterize the FAU faculty to this day.

The university's first students could pursue bachelor's degrees in five colleges: Business, Education, Humanities, Science and Social Science.  The College of Education also offered master's degrees in elementary, secondary and higher education, administration, guidance, special education and human behavior.

In April 1965, just seven months after opening its doors, FAU held its first commencement ceremony, presenting degrees to 30 students who had entered as seniors.  Because there was no appropriate facility on campus for this event, the ceremony was conducted at the First Presbyterian Church of Boca Raton.

The aggressive construction program that took place during the university's earliest years saw completion of the three-story Administration Building (which three decades later would be named for Dr.\ Williams), the Humanities Building, which includes the 504-seat University Theater, and six residence halls, all named to honor Native American tribes: Algonquin, Modoc, Mohave, Naskapi, Sekoni and Seminole.

In the fall of 1965, FAU introduced the nation's first degree program in ocean engineering.  Over the years, this pioneering program has garnered much recognition, including being named a State University System Program of Distinction.  Today it is housed at SeaTech, a state-of-the-art research center in Dania Beach, and offers bachelor's, master's and doctoral degrees.

Intercollegiate athletics made their appearance at FAU in 1969, signaling the start of the gradual transformation of the university into a more traditional institution than was originally envisioned.  This trend continued in later years with the advent of fraternities and sororities, an annual Homecoming celebration and construction of the University Center, which quickly became a hub of student life.

Every university must have a mascot, and FAU found one on its doorstep: the feisty burrowing owl, a South Florida native that lives and raises its chicks in holes in the ground.  Classified as a species of special concern by the Florida Fish and Wildlife Commission, the owls dwelling on FAU's campuses have long been protected from human harassment.  In 1971, the Boca Raton campus became an official burrowing owl sanctuary.  The university's teams proudly bear the Owls name.

During Dr.\ Williams' presidency, major emphasis was placed on developing the Boca Raton campus, but some outreach efforts were made to other parts of FAU's large, seven-county service area.  Small satellite facilities were established in Fort Lauderdale to the south and West Palm Beach and Fort Pierce to the north.

When Dr.\ Williams retired in 1973, he left a university that had made great progress under his leadership. The student body had increased to 5,632, the number of degree programs had expanded from 31 to more than 100, and 13,509 men and women had graduated to become FAU alumni.  The stage was set for the next phase in the university's development.

\section{The Creech Years (1973--1983)}

President Glenwood L.\ Creech came to FAU from the University of Kentucky, where he had been vice president of university relations.  A courtly Southern gentleman with wavy salt-and-pepper hair and a movie star smile, Dr.\ Creech was ideally suited to tackle the urgent challenge of increasing financial support for the university.  To encourage substantial private donations, the state had introduced a program that would match every gift of \$600,000 made to endow an Eminent Scholar Chair with \$400,000 in state funds, boosting the value of the donation to \$1 million.  Dr. Creech used this leveraging tool with great success, and FAU soon became the state leader in the procurement of these endowed chairs.

Million-dollar Eminent Scholar Chairs established under Dr.\ Creech include the Charles E.\ Schmidt Chair in Engineering, the Dorothy F.\ Schmidt Chair in the Performing and Visual Arts, the Charles Stewart Mott Chair in Community Education, the Eugene and Christine Lynn Chair in Business and the Robert J.\ Morrow Chair in Social Science.

The Schmidt and Lynn families were to become sustaining friends of the university, demonstrating real interest in its development over the years and making multiple donations of astounding generosity.  By 2001, the Schmidts had contributed more than \$53 million to FAU, including state matching funds, enriching the life of the university in a host of ways, from establishing an innovative medical education partnership with the University of Miami to attracting legends of the American theatre to FAU's performing arts program.  Occupants of the Dorothy F.\ Schmidt Chair in the Performing Arts have included director Joshua Logan, Pulitzer Prize-winning playwright Edward Albee, and Tony Award-winning actor Hume Cronyn.  The Schmidt Family Foundation's 1998 gift of \$15 million (\$30 million with the dollar-for-dollar state match permitted at that level) set a record for private donations to public education in Florida.  Today the College of Arts and Letters bears the name of Dorothy F.\ Schmidt and the College of Science that of her husband, Charles.

Eugene and Christine Lynn focused their philanthropy on the College of Business and the College of Nursing, donating more than \$32 million over the course of two decades.  A former registered nurse, Mrs. Lynn made a \$10 million (\$20 million with the state match) to FAU's widely admired nursing program in 2001.  The College of Nursing is named in her honor.

During the presidency of Dr.\ Creech, the campus began to take on a new look, thanks to his success in getting the state and private parties to donate landscaping to the largely barren former airfield.  He asked for and received a \$5,000 grant from Tallahassee to plant trees on campus, and he invited the community to help in the beautification effort.  Ever the FAU supporter, Tom Fleming responded to the call with six huge ficus trees, which provided deep wells of shade on the lawn in front of the Administration Building until they were uprooted during the hurricanes of 2004 and 2005.  In the mid-1970s, mathematics professor Jack Freeman organized a work party of students that, with some help from the Florida Department of Transportation, managed to carry out the Herculean task of digging up and moving several dozen full-grown live oak trees from the path of I-95, which was under construction a half-mile west of the Boca Raton campus.  These trees took root in several spots, most notably at the south end of the Breezeway where they stand today as Heritage Park.

At the end of his decade in office, Dr.\ Creech could take justifiable pride in a university that had matured both academically and physically under his leadership.  Major additions to campus included the University Center and its 2,400-seat auditorium, the Engineering Building and the 70,000-square-foot Gymnasium.  As a tribute to Dr.\ Creech upon his retirement in 1983, donors funded the Glenwood and Martha Creech Eminent Scholar Chair in Science.  That year, as it approached its 20th anniversary, the university had 9,388 students and its alumni base had grown to 30,243.  Some big changes lay ahead.

\section{The Popovich Years (1983--1989)}

In July 1983, FAU welcomed the first woman to head a public university in Florida history, Dr.\ Helen Popovich. For the 18-month period preceding her appointment, she had been the acting president of Winona State University in Minnesota.

One far-reaching change implemented at FAU during her presidency was the addition of freshman and sophomore classes to the student body in 1984.  South Florida's rapidly expanding population had generated strong demand for a four-year university, and FAU answered the call.  The university simultaneously maintained its cooperative relationship with all of the community colleges in its service region, tailoring ``2+2'' degree programs that allowed community college graduates to move on to university studies without facing transitional obstacles.

Dr.\ Popovich placed special emphasis on adding more minorities to the university's faculty, staff and student body, and she succeeded in pushing those numbers upward. During her six years in office, the enrollment of African American and Hispanic students increased, and the faculty and administration became more diverse. She also supported the appointment of more women to faculty and administrative positions, and she encouraged creation of the Women's Studies certificate program.

FAU's ability to serve Broward County students in their home community took a significant step forward with the opening of the Reuben O'D.\ Askew University Tower in downtown Fort Lauderdale in 1987.  This nine-story classroom and office building, named after a former Florida governor who later joined the faculty for a short period of time, housed programs in business and public administration.

Graduate programs and research activity also advanced during Dr.\ Popovich's administration.  In 1989, her last year in office, FAU reached the important milestone of \$10 million in sponsored research.

When Dr.\ Popovich left FAU to accept the presidency of Ferris State University in Big Rapids, Michigan, enrollment stood at 11,743 and alumni had increased to 41,152.

\section{The Catanese Years (1990--2002)}

With the arrival of Dr.\ Anthony J.\ Catanese in January 1990, Florida Atlantic University entered a period of rapid growth and development on all fronts.  During his presidency the student body more than doubled, four new campuses were built, three dozen new degree programs were introduced and the Owls  ---  including a football team that made its debut in 2001  ---  began competing in Division I of the NCAA.  By the middle of Dr.\ Catanese's 12-year term, FAU had become known as the fastest growing university in America, and that was probably literally true.

Formerly dean of the College of Architecture at the University of Florida, Dr.\ Catanese became president of FAU just as higher education in Florida and around the United States began to get caught in the crossfire of a recessionary economy and changing national priorities.  The central challenge he and other educators across America faced was to do more with less: the demand for higher education was soaring as traditional public funding sources were contracting.  Despite these challenges, Dr.\ Catanese embraced an ambitious vision for FAU, driven by his belief that universities had to �run smarter� by adopting some of the principles of private enterprise, including putting productivity standards in place, containing expenses and seeking resource-leveraging partnerships.  He succeeded perhaps beyond even his own wildest dreams, presiding over a half-billion-dollar construction program on seven campuses that created more than one million square feet of new and renovated classroom, laboratory and office space.

As the university's student body passed the 23,000 mark, its faculty expanded to include 895 full-time, tenure-track teachers and researchers, and its degree offerings increased to 137, FAU built new campuses in Davie, Dania Beach, Jupiter and Port St.\ Lucie.  The Downtown Fort Lauderdale campus was greatly enhanced by the addition of the Florida Atlantic University/Broward Community College Higher Education Complex, a 12-story high-tech facility.

FAU's sponsored research activity increased from \$10 million to \$37 million annually, and the 52-acre Florida Atlantic Research and Development Park took shape on the Boca Raton campus.

Dr.\ Catanese led the first capital campaign in the university's history, and this, too, was highly successful, increasing the assets of the FAU Foundation from \$18 million to \$150 million.

Especially notable accomplishments of the Catanese years include creation of FAU's medical education program, in partnership with the University of Miami; introduction of a five-year professional degree program in architecture, based at the Downtown Fort Lauderdale campus; and establishment of the four-year, residential Harriet L.\ Wilkes Honors College on the university's John D.\ MacArthur campus in Jupiter.

When Dr.\ Catanese left FAU in 2002 to become president of the Florida Institute of Technology in Melbourne, the student body had grown to 23,836 and the university had 78,396 alumni.  Through aggressive recruitment of minorities, FAU's student body had become the most diverse in Florida's State University System, with African Americans making up 16 percent and Hispanics 13 percent of the total number of students in attendance.

\section{The Brogan Years (2003-- )}

On January 31, 2003, by unanimous vote of the university's Board of Trustees, Frank T.\ Brogan --- Florida's incumbent lieutenant governor and a 1981 graduate of FAU's master's degree program in educational administration --- was named the fifth president of Florida Atlantic University.

President Brogan's return to his alma mater as its president marked the latest development in his career in public education in Florida, which began in 1978 when he became a teacher at Port Salerno Elementary School in Martin County.  After serving as a teacher and administrator for 10 years, he was twice elected superintendent of schools in Martin County.  In 1995, voters around the state sent him to Tallahassee as commissioner of education.  He was elected Florida's lieutenant governor in 1999 and 2003, leaving the state's second-highest post early in his second term to accept the presidency of FAU.

President Brogan took the reins of an institution that bore little resemblance to the university that had opened its doors on an abandoned airfield in 1964.  By 2003 FAU was able to offer students a selection of 60 bachelor's degree programs, 53 master's degree programs, three specialist's degree programs and 17 doctoral degree programs on seven well-designed and beautifully maintained campuses.  Modern residence facilities were available on the Boca Raton and Jupiter campuses, and FAU students were being served through eight colleges: the Dorothy F.\ Schmidt College of Arts \& Letters, the Charles E.\ Schmidt College of Science, the Christine E.\ Lynn College of Nursing, the Harriet L.\ Wilkes Honors College, and the Colleges of Business, Education, Engineering, and Architecture, Urban \& Public Affairs.  Additional educational programs were offered by Open University \& Continuing Education and the Lifelong Learning Society.  In fulfillment of the original vision of the university's founders, FAU finally was able to make extensive use of distance learning technology, delivering many courses online and via videotape.  By 2002, the number of students taking advantage of distance learning opportunities had reached 16,000.

In the aftermath of FAU's 12 years of rapid expansion, President Brogan placed his focus on bringing added depth and quality to all of the university's programs and services.  He and his wife, Courtney Strickland Brogan, became the first presidential couple to live in the Eleanor R.\ Baldwin House, the newly built president's residence on the Boca Raton campus.  Two years after their arrival, the FAU family would grow in a way it never had before as the Brogans welcomed their newborn son, Colby John.  From his earliest days Colby John was in attendance at such major university events as commencement ceremonies and the Fall Family Festival, where he joined his parents at the head of a campus parade in the presidential golf cart, ``Owl Force One.''

In accepting the challenge of leading his alma mater, President Brogan said\relax
%
\footnote{The author has modified original text here to demonstrate that both quotations and footnotes are typeset single-spaced.  The same holds true for figure captions, although this has not been illustrated explicitly.}\relax
%
,
%
\begin{quotation}
I am humbled and thrilled to be selected FAU's fifth president. My lifetime of public service and commitment to public education continues as we all work together to lift FAU to the next level of excellence.
\end{quotation}
%
That mission got off to an impressive start in October 2003 when it was announced that the internationally known Scripps Research Institute of LaJolla, California, had decided to open an East Coast center of operations in Palm Beach County, and FAU had been selected to be the research giant's first university partner in Florida.  This alliance with the world's largest non-profit biomedical research organization offered ``boundless research, faculty and student partnership opportunities,'' President Brogan said.  It came along at an especially fitting moment in the university's history, since the State of Florida had recently awarded FAU \$10 million to establish the Center of Excellence in Biomedical and Marine Biotechnology, a research facility dedicated to searching Florida's coastal waters for sources of new pharmaceuticals that could be used to treat cancer, heart disease and other serious illnesses. The mission of Scripps and this new center dovetailed perfectly.

The Scripps announcement triggered wave upon wave of unbridled enthusiasm among Florida's lawmakers, who could see the dawning of a whole new day for the state's traditionally agriculture- and-tourism-dependent economy.  Gov.\ Jeb Bush likened the arrival of Scripps in Palm Beach County to the opening of Disney World in Orlando and signed into law a \$310 million package of financial incentives to seal the deal.  The Palm Beach County Commission sweetened the pot with \$200 million in additional funding.  As 2003 came to a close, FAU announced plans to house Scripps scientists in 10,000 square feet of laboratory space on the Boca Raton campus pending construction of a temporary facility to be built especially for Scripps on the Jupiter campus.  With great fanfare, ground was broken for the new \$12 million research building in February 2004.

The arrival of Scripps was the linchpin in a trend toward partnerships that had begun at FAU in the 1990s as a way of increasing the university's involvement with outside organizations and attracting new streams of funding to support the university's activities, particularly in the research arena.  Earlier partnerships had been concluded with the University of Miami Miller School of Medicine, Boca Raton Community Hospital, the North Broward Hospital District, the Harbor Branch Oceanographic Institution and the Smithsonian Marine Station, among other major entities.  In subsequent years, that list would grow to include three more prominent biomedical research organizations: the Torrey Pines Institute for Molecular Studies, based in San Diego, California, Tampa's H.\ Lee Moffitt Cancer Center \& Research Institute and Germany's internationally renowned Max Planck Society for the Advancement of Science.

These strategic partnerships were part of a cascade of developments that brought FAU unprecedented recognition in the area of biomedical research.  Especially outstanding achievements included Dr.\ Herbert Weissbach's work, in conjunction with researchers from the University of Iowa, Cornell University, the University of Pennsylvania and Germany's Friedrich Schiller University, on the role a particular enzyme might play in treating patients suffering from heart and brain diseases, and Dr.\ Ramaswamy Naraynan's patenting of a noninvasive, gene-based method of detecting colon cancer.

Significant research was also under way in the College of Engineering and Computer Science. Dr.\ William Glenn, one of the nation's foremost optical engineers with more than 150 patents to his credit, moved forward with the development of the world's most advanced high-definition television camera.  His work was funded by NASA, with a view toward placing the camera aboard space shuttles and the international space station.  Dr.\ Mike Lin, the Charles E.\ Schmidt Eminent Scholar in Engineering, received a National Science Foundation grant to identify why artificial heart valves sometimes fail, triggering life-threatening emergencies.  And at FAU's SeaTech ocean engineering research center in Dania Beach, the U.S. Navy continued its long-term funding commitment to the development of autonomous underwater vehicles, which have the ability to search for mines and carry out other military missions without putting human lives in jeopardy.

As FAU's student body continued to grow and its research and community engagement activities escalated, the university's economic impact increased dramatically.  In 2004 the Office of Institutional Analysis and Effectiveness released a report showing that FAU's economic impact on Palm Beach, Broward and St.\ Lucie counties exceeded \$1 billion annually.  By this time, the university ranked as the largest employer in South Palm Beach County, with more than 4,000 full-time and part-time employees on board, including about 1,500 faculty members.

Throughout the 2004-05 academic year, the university celebrated the $40^{\mathrm{th}}$ anniversary of the day it opened its doors to students in 1964.  Festivities got under way in October with a luncheon for founding faculty and staff members.  Retirees came from near and far to attend the luncheon and hear President Brogan describe them with admiration as ``the founding fathers and mothers of FAU.''

The university's $40^{\mathrm{th}}$ anniversary year began with assaults by the first strong hurricanes to hit South Florida since Hurricane Cleo delayed the opening of FAU by six days in the fall of 1964.  Hurricanes Frances and Jeanne caused millions of dollars in damage to the Broward, Boca Raton, Jupiter and Treasure Coast campuses and destroyed the homes of some students, faculty and staff members.  A relief effort was immediately undertaken by the Division of Student Affairs and the Daniel B.\ Weppner Volunteer Center, resulting in the donation of money, clothing and supplies for members of the FAU community who needed help getting back on their feet.  The Caribbean island nation of Haiti, home to many FAU students, was especially hard hit by Hurricane Jeanne.  In the immediate aftermath of the storm, some 400,000 people were homeless.  Konbit Kreyol, FAU's Haitian student organization, spearheaded a drive to send relief supplies to Haiti through the Deerfield Beach-based non-profit organization Food for the Poor.

Major facilities construction, which had moved forward at an astounding pace throughout the 1990s and into the opening years of the 21st century, continued to transform FAU's campuses.  New buildings to house classrooms, laboratories and offices that opened their doors between 2003 and 2008 included, on the Boca Raton campus, the Christine E.\ Lynn College of Nursing; the Louis and Anne Green Memory and Wellness Center; the Marleen and Harold Forkas Alumni Center; Indian River Towers, Heritage Park Towers and Glades Park Towers (state-of-the-art student residence halls); the DeSantis Pavilion, the Sean Stein Pavilion and the Office Depot Center for Executive Education in the Barry Kaye College of Business; and the Paul C.\ Wimbish Wing of the S.E.\ Wimberly Library; on the Davie campus, the Student Union; on the Jupiter campus, the FAU/Scripps Joint-Use Research Facility, the Harriet L.\ Wilkes Psychology Building, a new library, a new classroom building and the Maltz Lifelong Learning Center; on the Port St.\ Lucie campus, the Phase II Expansion Building; and at Harbor Branch in Fort Pierce, the Marine Science Partnership Building.  Making good use of FAU's expanding facilities was a constantly growing student body, which surpassed the 26,000 mark in 2007.

As gratifying as this rapid growth was, President Brogan and the Board of Trustees recognized that it needed to be guided by a carefully thought-out strategic plan. The planning effort began in earnest in 2004 with the formation of a task force led by Trustee Nancy Blosser and Jupiter Campus Vice President Kristen Murtaugh.  The plan aimed to achieve four goals that aligned with those that had been adopted on a statewide basis by the Florida Board of Governors, plus three additional goals formulated to meet specific needs that existed at FAU.  The goals of the university's 2006--2013 strategic plan are:
%
\begin{enumerate}
\item Access to and production of degrees
\item Meeting statewide professional and workforce needs
\item Building world-class academic programs and research capacity
\item Meeting community needs and fulfilling unique institutional missions
\item Building a state-of-the-art information technology environment
\item Enhancing the physical environment
\item Increasing the university's visibility
\end{enumerate}
%
Development of the plan took about 18 months and benefited from the input of many people, including faculty, staff, students, alumni and members of the outside community.  It was adopted by unanimous vote of the Board of Trustees on January 18, 2006.

As people around the country and the world became increasingly preoccupied with the threat of global warming, FAU sought ways to become part of the solution. The university's first ``green'' building was the headquarters facility of the Christine E.\ Lynn College of Nursing on the Boca Raton campus, which met the ``gold'' standard of the U.S.\ Green Building Council.  Shortly after that certification was received, FAU announced plans to build a new home for the College of Engineering and Computer Science at the ``platinum'' level and a new elementary school at the ``silver'' level.  The K-5 environmental magnet school, located on the grounds of FAU's Pine Jog Environmental Education Center in West Palm Beach, became the first public elementary school in Florida designed to meet ``green'' standards.  The university made public its intention to have every new facility achieve at least the ``silver'' level of certification.

In 2007, President Brogan signed the American College and University Presidents' Climate Commitment, which brought FAU into the national campaign to reduce greenhouse gas emissions by 80 percent by the middle of the $21^{\mathrm{st}}$ century.  That was followed by establishment of the university-wide ``Mission Green'' campaign, a broadly focused initiative aimed at promoting environmental sustainability through changes in operating procedures, education and community outreach.  A Campus Sustainability Committee that included representation by students, faculty and staff was created to address issues related to FAU's global environmental footprint.

The university began to play a high-profile role in the search for clean, affordable energy in 2006, when the state awarded FAU \$5 million to establish a second Center of Excellence.  This facility, housed at the SeaTech ocean engineering research center on Dania Beach, embraced the mission of harnessing the power of Florida's strong offshore currents, particularly the Gulf Stream, to generate massive amounts of electricity.

Throughout President Brogan's first six-year term, the university grew in ways both conventional and innovative.  FAU High School, a dual enrollment program for students with exceptional academic abilities and strong self-motivation, opened in 2004.  Two years later, the Charles E.\ Schmidt College of Biomedical Science was established to provide a home for FAU's medical education program.  That was followed by creation of the Graduate College to provide a single resource center for all graduate programs university-wide.  In 2008, FAU merged with the Harbor Branch Oceanographic Institution (HBOI) in Fort Pierce, a leader in marine science and related fields.  As a unit of FAU, HBOI changed its name to the Harbor Branch Oceanographic Institute.

In a multitude of ways, FAU was becoming a force to be reckoned with on both the regional and national levels.  This extended to the realm of athletics, as FAU's 18 NCAA Division I teams continued to excel on the field of competition.  In 2007 FAU's seven-year-old football team, led by legendary coach Howard Schnellenberger, became the youngest team in NCAA history to be invited to a bowl game.  The Owls proved themselves worthy of this honor by beating the University of Memphis 44--27 in the New Orleans Bowl and then duplicating the feat in 2008 by defeating Central Michigan 24--21 in the Motor City Bowl.  These victories gave the Owls the distinction of becoming the only football team in Florida to win back-to-back bowl games in 2008.

In the fall of 2008, as FAU embarked upon its $45^{\mathrm{th}}$ year of service, enrollment stood at 27,000 students.  More than 170 bachelor's, master's, specialists and doctoral degree programs were available to them.  With more than 43 percent of its student body classified as minority or international, FAU ranked as the most ethnically and culturally diverse institution in Florida's State University System.  In addition to serving its regularly enrolled student body, the university offered education and enrichment to 20,000 men and women of retirement age through the largest university-based lifelong learning program in the United States.  The university's expanding research portfolio earned FAU a designation of ``High Research Activity'' from the Carnegie Foundation for the Advancement of Teaching.  With more than 3,000 employees and a regional economic impact in excess of \$1.1 billion annually, FAU had become a strong engine of economic growth.  Citing his dedicated and effective leadership, in September 2008 the FAU Board of Trustees unanimously awarded President Brogan a second six-year contract, to begin in March 2009.


\chapter{An Empty Chapter With a Very Long Title Included Solely to See How It Looks in the Table of Contents}

Here is some text for this empty chapter.  We need to finish one line to see how it lines up.  There, that ought to do it.

Here is some more text, which is included to see how subsequent paragraphs will look.  There, that ought to do it.

\section{An Empty Section With a Very Long Title Included Solely to See How It Looks in the Table of Contents}

Here is some text for this empty section.  We need to finish one line to see how it lines up.  There, that ought to do it.

Here is some more text, which is included to see how subsequent paragraphs will look.  There, that ought to do it.


\begin{figure}[b]
\begin{center}
\parbox{1.0\columnwidth}{\hrulefill\quad1.0\quad\hrulefill}
\parbox{0.8\columnwidth}{\hrulefill\quad0.8\quad\hrulefill}
\parbox{0.6\columnwidth}{\hrulefill\quad0.6\quad\hrulefill}
\parbox{0.4\columnwidth}{\hrulefill\quad0.4\quad\hrulefill}
\end{center}
\setlength\capwidth{0.6\textwidth}
\caption[A sample figure]{This figure has been included to show how floating objects work.  It should appear at the bottom of the page.}
\end{figure}


\begin{figure}[b]
\begin{center}
\parbox{1.0\columnwidth}{\hrulefill\quad1.0\quad\hrulefill}
\parbox{0.8\columnwidth}{\hrulefill\quad0.8\quad\hrulefill}
\parbox{0.6\columnwidth}{\hrulefill\quad0.6\quad\hrulefill}
\parbox{0.4\columnwidth}{\hrulefill\quad0.4\quad\hrulefill}
\end{center}
\caption[Another sample figure]{This figure has been included to make sure the caption width reverts to its correct, default value.  It too should appear at the bottom of the page.}
\end{figure}

\subsection{An Empty Subsection With a Very Long Title Included Solely to See How It Looks in the Table of Contents}

Here is some text for this empty subsection.  We need to finish one line to see how it lines up.  There, that ought to do it.

Here is some more text, which is included to see how subsequent paragraphs will look.  There, that ought to do it.

\subsubsection{An Empty Subsubsection With a Very Long Title Included Solely to See How It Looks in the Table of Contents}

Here is some text for this empty subsubsection.  We need to finish one line to see how it lines up.  There, that ought to do it.

Here is some more text, which is included to see how subsequent paragraphs will look.  There, that ought to do it.

\paragraph{An Empty Paragraph With a Very Long Title Included Solely to See How It Looks in the Table of Contents}

Here is some text for this empty paragraph.  We need to finish one line to see how it lines up.  There, that ought to do it.

Here is some more text, which is included to see how subsequent paragraphs will look.  There, that ought to do it.

\subparagraph{An Empty Subparagraph With a Very Long Title Included Solely to See How It Looks in the Table of Contents}

Here is some text for this empty subparagraph.  We need to finish one line to see how it lines up.  There, that ought to do it.

Here is some more text, which is included to see how subsequent paragraphs will look.  There, that ought to do it.


\chapter{A Chapter With the Chemical Formula%
	\texorpdfstring{\\ $\rm Fe_2 O_3$}{Fe2O3} In Its Title}

Some text before the first section.

\section{A Section With the Chemical Formula 	\texorpdfstring{$\rm Fe_2 O_3$}{Fe2O3} In Its Title}

Some text in the first section.


\chapter{An Extra Chapter to Get The Appendices to Appear on the Second Page of the Table of Contents}

This space was intentionally left blank.  Oh, wait.  No, it wasn't.



\backmatter


\chapter{A Test of the Appendix System}

This is only a test.  Repeat, this is only a test.

\section{A Section Within an Appendix}

This is only a test.  Repeat, this is only a test.

\subsection{A Subsection Within an Appendix}

This is only a test.  Repeat, this is only a test.


\chapter{Another Test of the Appendix System}

This is only another test.  Repeat, this is only another test.



\begin{thebibliography}{M}

\bibitem{HFAU}
	Florida Atlantic University official web site.
	``History of Florida Atlantic University.''
	\texttt{http://www.fau.edu/explore/history.php}.
	Cited 18 January 2010.

\bibitem{test1}
	A.O.~Burroughs.
	A quite learned article.
	\textit{A Very Learned Journal}	{\boldmath $n$} (2010) a few pages.

\bibitem{test2}
	A.O.~Burroughs.  
	A really very learned article.
	\textit{Another Very Learned Journal} {\boldmath $n+1$} (2010) several pages.

\bibitem{test3}
	A.O.~Burroughs.  
	An extremely learned article on which this thesis is mostly based.
	\textit{A Very Prestigious Journal} {\boldmath $n^2$} (2010) many pages.

\end{thebibliography}

\end{document}  